\documentclass[chaptersright]{informeutn}
\usepackage[utf8]{inputenc}
\usepackage{array}
\usepackage{geometry}
\usepackage[table]{xcolor}
\usepackage{colortbl}
\usepackage{caption}
\usepackage{graphicx}
\usepackage{amsmath}
\usepackage{multirow}
\usepackage{float}

\materia{Dispositivos Electronicos I}
\titulo{Trabajo Practico N°5: TIRISTORES}
\comision{3R2}
\autores{Documentador y operador: Angelo Prieto 401012\\ Coordinador: Gaston Grasso 401892}
\fecha{14-10-2025}


\begin{document}
\maketitle
\tableofcontents

\chapter{Introduccion}

\chapter{SCR}
\section{Condición de disparo y corriente de mantenimiento}
\subsection{Actividad de laboratorio}
\subsection{Actividad de simulación}
\section{Obtención de curva característica}
\subsection{Actividad de laboratorio}
\section{Funcionamiento con corriente alterna}
\subsection{Actividad de simulación}

\chapter{DIAC}
Objetivo: determinar la polarización y funcionamiento del DIAC.
\section{Actividad de laboratorio}

\chapter{TRIAC}
\section{Polarización y funcionamiento}
\subsection{Actividad de laboratorio}
\section{Aplicación: control de disparo (Dimmer)}
\subsection{Actividad de laboratorio}

\chapter{Interpretación de las especificaciones del fabricante}

\chapter{Conclusión}
\end{document}
