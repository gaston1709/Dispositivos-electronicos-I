\documentclass{informeutn}

\materia{Dispositivos Electrónicos}
\titulo{Trabajo práctico N°6: Flujo de diseño analógico con CMOS}
\comision{3R2}
\autores{Coordinador: Prieto Angelo 401012 \\
        Documentador y operador: Gaston Grasso 401892}
\fecha{30-11-2025}

\begin{document}
\maketitle
\tableofcontents

\chapter{Introducción}
En el presente trabajo práctico se aborda el flujo completo de diseño analógico utilizando 
tecnología CMOS, con el objetivo de comprender y aplicar las etapas necesarias para 
desarrollar un circuito integrado desde su especificación funcional hasta su implementación 
física.

Para ello, se emplea el proceso tecnológico SKY130, un Process Design Kit (PDK) de libre 
acceso que permite realizar todas las etapas del diseño, desde la simulación eléctrica del 
circuito hasta la verificación del layout respetando reglas de diseño reales. Asimismo, se 
utilizan herramientas opensource de diseño de circuitos integrados, con el fin de validar 
esquemáticos, realizar simulaciones y asegurar la correctitud física mediante 
verificaciones DRC (Design Rule Check) y LVS (Layout Versus Schematic).

El circuito elegido para este trabajo es un inversor analógico CMOS, una de las celdas más 
básicas y esenciales en el diseño digital y analógico. Su estructura, compuesta por un 
transistor PMOS y un transistor NMOS en configuración complementaria, permite obtener un 
comportamiento eficiente y robusto, lo que lo convierte en un punto de partida ideal para 
comprender el flujo de diseño y las particularidades de la tecnología CMOS.

A lo largo del informe se describen las etapas seguidas en el desarrollo del inversor: el 
análisis y simulación del esquemático, la implementación del layout siguiendo las reglas 
específicas del proceso, y las verificaciones necesarias para asegurar la fabricabilidad 
del diseño. De esta manera, se integra el conocimiento teórico estudiado con herramientas 
prácticas modernas, brindando una experiencia completa y representativa del flujo de diseño 
utilizado en la industria de circuitos integrados.

\chapter{Actividad 1: especificaciones de diseño}

El inversor CMOS a diseñar debe cumplir las siguientes especificaciones:\\
Tecnología: SKY130 (130 nm CMOS).\\
Voltaje de alimentación (VDD): 1.8 V.\\
Caracteristicas del transistor PMOS:
\begin{itemize}
    \item L = 0,15 (Longitud de canal del transistor en micrómetros)
    \item Wp = 2,1 (Ancho del canal del transistor en micrómetros)
    \item nf = 1 (Número de fingers)
    \item mult = 1 (Multiplicidad)
    \item model: pfet\_01v8
\end{itemize}
Caracteristicas del transistor NMOS:
\begin{itemize}
    \item L = 0,15
    \item Wn = 1,05
    \item nf = 1
    \item mult = 1
    \item model = nfet\_01v8
\end{itemize}
Ajustar Capacidad de carga de salida: 2 pF (considerando fan-out).\\
Tiempo de transición objetivo: menor a 50 ns para subida y bajada.\\

\section{Instalación de herramientas}

Para el diseño y simulación del inversor CMOS se utilizaron herramientas opensource compatibles 
con el PDK SKY130:
\begin{itemize}
    \item Xschem para diseñar y simular esquemáticos.
    \item Ngspice: modelos spice para simulación eléctrica en el entorno de Xschem.
    \item Magic para ruteo y layout, servirá ademas para verificación del diseño físico (DRC).
    \item Netgen para realizar la comparación LVS (Layout versus Schematic).
\end{itemize}

\chapter{Actividad 2: flujo de diseño analógico}
\begin{figure}[ht!]
    \centering
    \includegraphics[width=1\linewidth]{pictures/flujo-diseno.png}
    \caption{flujo de diseño.}
    \label{fig:flujo-diseño}
\end{figure}

Para esta actividad, se sigue cuidadosamente el procedimiento descrito en las consignas del 
trabajo.

\section{Schematic design}
Para este parte del diseño se ha utilizado el programa Xschem. Se puede observar el resultado 
obtenido en las figuras  \ref{fig:schem} (esqumático del inversor) y \ref{fig:sym} (símbolo 
diseñado para el inversor).

\begin{figure}[ht!]
    \centering
    \includegraphics[width=1\linewidth]{pictures/inversor esquematico.png}
    \caption{captura de pantalla en programa Xschem. Esquemático de inversor.}
    \label{fig:schem}
\end{figure}

\begin{figure}[ht!]
    \centering
    \includegraphics[width=1\linewidth]{pictures/inversor symbol.png}
    \caption{símbolo de inversor.}
    \label{fig:sym}
\end{figure}

\section{Simulation}
Para esta parte del diseño también se ha utilizado el programa Xschem. Se ha insertado el 
símbolo del inversor diseñado en un circuito, y se ha realizado las simulaciones pertinentes, 
comprobando así el correcto funcionamiento. Se puede observar el resultado en la figura 
\ref{fig:sim}.

\begin{figure}[ht!]
    \centering
    \includegraphics[width=1\linewidth]{pictures/simulacion esquematico.png}
    \caption{simulación de esquemático diseñado.}
    \label{fig:sim}
\end{figure}

\section{Layout: placement / routing}
Para esta parte del diseño se ha utilizado el programa Magic. Se puede observar el resultado obtenido en la figura \ref{fig:enrutado}.

\begin{figure}[ht!]
    \centering
    \includegraphics[width=1\linewidth]{pictures/chip expandido.png}
    \caption{enrutado de chip.}
    \label{fig:enrutado}
\end{figure}

\section{DRC}
Para esta etapa, se ha utilizado nuevamente el programa Magic, y se ha ejecutado el verificador
de reglas de diseño DRC. Se ha obtenido un resultado satisfactorio al no presentarse ningún error.
Se puede observar en la figura \ref{fig:drc}.

\begin{figure}[ht!]
    \centering
    \includegraphics[width=1\linewidth]{pictures/zero error DRC.png}
    \caption{captura DRC.}
    \label{fig:drc}
\end{figure}

\section{Post-layout}
Una vez finalizado el diseño físico del circuito, se realizaron simulaciones post-layout con el fin de evaluar el impacto de los efectos y parásitos introducidos durante el proceso de layout. Para ello, primero se efectuó la extracción de los elementos parasitarios, obteniendo un netlist que incluye las resistencias y capacitancias asociadas a las interconexiones y a la geometría final del diseño.

Posteriormente, este netlist fue combinado con los modelos del proceso del PDK Sky130, lo que permitió incorporar las características eléctricas reales de los dispositivos. Con esta información, se llevaron a cabo simulaciones en un entorno SPICE para analizar parámetros como retardos de propagación, integridad de señal y consumo de potencia.

Luego, se compararon los resultados post-layout con las simulaciones pre-layout basadas únicamente en el esquemático.

Finalmente se generó el archivo .gds, el cual se manda a la foundry para iniciar la producción del chip.

\chapter{Conclusión}

En conclusión, pudimos cumplir los objetivos del trabajo práctico: diseñamos el inversor CMOS en Xschem, lo simulamos, hicimos el layout en Magic y llegamos hasta generar el archivo \texttt{.gds} listo para enviar a la foundry. Eso nos permitió ver el flujo completo, desde el esquemático hasta la implementación en silicio usando el proceso SKY130.

En el camino aparecieron varias trabas con las herramientas opensource (configuración del PDK, integración entre programas, mensajes de error poco claros), pero justamente eso ayudó a entender mejor qué hace cada etapa del flujo de diseño de microelectrónica y a valorar todo el trabajo que hay detrás de un circuito integrado, incluso para una celda tan simple como un inversor.

\end{document}
